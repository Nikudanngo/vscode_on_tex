\documentclass[dvipdfmx,uplatex]{jsarticle}%A4縦用紙のおまじない
\newcommand{\bvec}[1]{\mbox{\boldmath $#1$}}%ベクトルを\bvec{文字}で表示できる
\usepackage{cancel}%\cancel{}で{}の中身に斜線を引ける
\usepackage{layout}%\layoutでレイアウトを表示する
\usepackage{siunitx}%si単位系を\si{\kilogram}=>kgのように表示できる
\usepackage[top=20truemm,bottom=20truemm,left=15truemm,right=15truemm]{geometry}%レイアウトをいじる
\begin{document}
\title{サンプル}
\author{えんどー}  
%\date{}とすれば日付が表示されない
\maketitle
\section{\LaTeX による論文の書き方}
\begin{itemize}
    \item ググる
    \large %これ以降文字を少し大きくする
    \item 書く
    \Large %これ以降文字を大きくする
    \item さらにググる
    \LARGE %これ以降文字をさらに大きくする
    \item さらに書く
    \small
    \item たまに休憩
\end{itemize}
\section{数式の書き方}
\large
    \begin{equation}
        F = ma
    \end{equation}
    \begin{eqnarray}
        e^{\bvec{i}\pi} + 1 = 0 \label{2} \\%\label{}で式番号を手動で割り当てる
        %\\で改行 πは\piで書く
        %通常の()はそのままでいいが大きい()は\left( 任意のテキスト \right)と書く
        e = \lim_{n \to \infty} \left( 1 + \frac{1}{n} \right) ^n \label{3} \\ %{eqnarray}を使えば二行書ける
    \end{eqnarray}
    \begin{eqnarray*} %*をつければ式番号が消える
        x &=& 1 + 2 \\ % &&で挟めば式をそろえられる
        x &=& 3 
    \end{eqnarray*}
\begin{enumerate}
    \item 新しい項目を作る
    \item[(a)] []のオプションを変更して項目の形式を指定
    \item[(1)] たかしくんはリンゴを握りつぶしてこう言いました$\cdots$  %文章中に\を使いたいときは$$で数式モードにする   
\end{enumerate}

\newpage  %ここで新しいページにする

\section{基本的に}
%{verbatim}で入力した通りに出力
    \begin{verbatim} 
        \関数名[オプション]{テキスト or 要素}で構成される
    \end{verbatim}
\end{document}